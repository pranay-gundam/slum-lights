\section{Conclusion}

This paper was made in effort to fill in gaps in the conventional understanding of slums which has been challenged in light of urbanization and slum growth trends. This paper makes two main contributions to existing literature. First, I introduce a novel approach to the spatial modeling of slums by using granular nighttime light satellite imagery and population density data. This modeling effort provides further input regarding qualitative characteristics of slums in these two aspects; specifically, I conclude that the log odds of a geo-location being considered a slum is low at high light output values and is high at a combination of low light output levels and high population density. This portion of the paper also paves the path for future research in predictive mapping of slums from publicly available data sets. Again, there are some assumptions built into the data processing which is a limitation that must be taken note of. Future work would be able to build off a smaller set of unverified assumptions and a more quality training data set as well. This paper used data from slum dwellers international but many other papers that discuss slums also create their own custom metrics for households living in slums that can be considered in the future. Finally, future work can experiment with more nuanced models such as other blackbox models like neural networks and pull training data from a wider net of countries.

The second contribution is a discussion of potential household choice utility functions that can be calibrated to understand what characteristics of a household would make living in a slum their optimal choice. I offer a very limited calibration which is likely restricted due to the quality of the data set on which it was calibrated on and the assumptions entailed with pricing slum rent. This portion of the paper contributes to a nascent field in urban economics that is analyzing the decision making that is involved in the formation of slums and how slum dwellers come to be. Future research into this field would require identifying more widespread, higher quality data sets for model calibration and further discussion into components that factor into a household's utility function for location choice.