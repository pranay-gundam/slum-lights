\section{General Equilibrium}

Our discussion thus far has been from an emperical partial equilibrium perspective. It is also a good exercise to consider the general equilibrium perspective with a granular spatial household choice model that incorporates slum choices as well. Fitting this model to data is difficult given the lack of availability so this section will be dedicated to setting up one such theoretical model. Given this thesis' adherence to granular spatial data the model will consider location blocks to be quite granular. Consider a static model of an economy with a set of $N$ location blocks. Each block $i\in [N]$ is endowed with some exogenous supply of land $H_i$ and amenities $A_i$. We define a slum dweller as a household who lives in some location block $i$ and faces residential land use $h_i$ such that $(h_i)^{\beta}(A_i)^{1-\beta} < \kappa$ where $\beta$ and $\kappa$ are city specific parameters to be estimated. 

\subsection{The Household Problem}

Households make decisions about where to live and face commuting costs $C_i$. To minimize the number of choices that households have to make (and the statespace of the resulting system), households only make a location choice as to where to live and every household commutes to some same exogenous working block that is not included in the model. One can think of this as a one-way commuting cost for each location that aggregates the difficulty of commuting to all other blocks and we can estimate this cost by defining some set of blocks in a city on google maps and taking the average commute time $\frac{1}{N}\sum_{j=1}^N C_{i,j}$ for each $i$. Households are also heterogenous in productivity and recieve wages based on a scaled value of this productivity. General household peferences take on a Cobb-Douglas form and are formed over living location block amenities ($A_i$), residential land use ($h_i$), and a consumption good ($c$). $$U_i = \left(\frac{c}{\alpha_1}\right)^{\alpha_1}\left(\frac{h_i}{\alpha_2}\right)^{\alpha^2}\left(\frac{A_i}{\alpha_3}\right)^{\alpha_3}$$ where $\alpha_1 + \alpha_2 + \alpha_3 = 1$. Individual residential land use depends on the population share living in location block $i$ and its corresponding land endowment $H_i$.

\subsection{Government Channels}
This model doesn't structurally incorporate a government but the main policy counterfactual of interest is changing location amenities or specific commuting costs; we can calculate the changes in population shares in slum areas as a function of these components of the model.
