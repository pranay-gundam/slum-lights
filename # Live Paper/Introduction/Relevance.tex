\section{Relevance}

Although studying how slums comes to form is an interesting question in itself there are also a few reasons as to why we should be concerned with the formation and characterization of them beyond trying to find an explanation for the decline of slums despite urban populations increasing and unmatched spending on infrastructure. Even if one argues that some may prefer urban poverty to rural poverty there are several unique characteristics that slums pose that make them unique relative to the general case of lower income neighborhoods. Marx, Stoker, and Suri\textsuperscript{\cite{Marx}} explain that slums act as a poverty trap for the majority of their residents. They \begin{quoting}document how human capital threshold effects, investment inertia, and a policy trap may prevent slum dwellers from seizing economic opportunities offered by geographic proximity to the city.\end{quoting}

Slum mapping and prediction has been a relatively new area of interest. Some papers like Friesen, Rausch, Pelz, and Furnkranz\textsuperscript{\cite{wdislums}} try to analyze national indicators that are most correlated with urban slum populations using data from the World Bank and modeling methods such as Random Forest, JRIP (rule learning algorithm), and J48 (decision tree learner). Their data set, however, is not minute enough to make any meaningful claims. Using country-level effects to predict city-level indicators confounds many variables and can paint a wrong picture especially in countries in which the disparity between cities is large. On the other hand, new satellite imagery is allowing researchers to look at very small details that allow for a more micro level analysis. Image recognition software is being used to characterize photos with slums in order to have a more clear global pictures. Issues arise, however, in the accessibility and quantity of these high resolution satellite images.

Mapping and identification are critical to ensuring that potential future policy that targets slums is accessible to all the areas that need assistance. Marx, Stoker, and Suri\textsuperscript{\cite{Marx}} for example recommend a paradigm for policies to address the growth of slums but in order to begin any sort of action one must have proper documentation as a foundation for any concrete plans. In addition, although slums may be the symptom of greater forces at play rather than the disease itself, learning more about how they form and why an individual chooses to live in a slum is crucial to understanding what of those greater forces need attention.

Exploring slum growth contributes a lot to understanding key factors in development of cities and developing a micro-model on what pushes households to live in slums is a good step in learning about the formation of clusters of housing and neighborhoods. Discrete choice models for housing location is a well-established field\textsuperscript{\cite{Mackay}} but research on the specific slum related implications of these discrete choice models is relatively sparse and a new field. Studies are focused on data in the form of sample size limited household surveys from an individual city to bypass the issue of a lack of widely available data of the same sort\textsuperscript{\cite{Badmos},\cite{Celhay},\cite{Deeyah},\cite{Roy},\cite{Das}}.


The following work will be in two parts. The first to explore new avenues for mapping slums, particularly using a novel data set: the Defense Meteorological Satellite Program Operational Line Scanner (DMSP/OLS) Nighttime Light Image Timeseries from NOAA. The second part of this paper calibrates a basic discrete choice model that takes a micro economic glance at what factors drive households to live in slums. While we have a clue about the general correlation at a macro level of what motivates slum growth, identifying the systematic decision bias that is the threshold between choosing to live in a slum or not can help stimulate targeted policy action and give more insight as to what is driving down slums proportions despite an increase of urban populations.
